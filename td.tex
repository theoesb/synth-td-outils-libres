\documentclass[12pt]{article}

% La gestion des fontes, sachant que ce document est sensé
% être compilé au moyen de XeLaTeX
%\usepackage{fontspec,xltxtra}

% Pour permettre la redéfinition des dimensions
\usepackage[a4paper]{geometry}

% Pour le support du français
\usepackage[francais]{babel}

% Les polices à utiliser
%\newcommand{\fontsphil}{
 % \defaultfontfeatures{Mapping=tex-text}
  %\defaultfontfeatures{Ligatures=TeX}
  %\setmainfont{Linux Libertine O}
  %\setsansfont{Liberation Sans}
  %\setmonofont[Scale=.9]{Ubuntu Mono}
  %\newfontfamily{\hwfont}{Comic Sans MS}
  %\newfontfamily{\fatfont}{Arial Black}
  %\newfontfamily{\authorfont}{Liberation Serif}
  %\newfontfamily{\spacefont}{Liberation Mono}
  %\fontspec[SmallCapsFont={Arial Black}]{Arial Black}
%}
%\fontsphil

% Des polices pour quelques caractères spéciaux
\usepackage{wasysym}
\usepackage{pifont}

% Gestion des images
\usepackage{graphicx}

% Gestion des couleurs
\usepackage{color}
\definecolor{fondsection}{rgb}{.9,.9,.9}

% Gestion des espaces
\usepackage{xspace}

% Gestion des URL et liens hypertexte
\usepackage{hyperref}

% Configuration des dimensions
\geometry{%
  left=20mm,width=165mm,
  top=15mm,height=267mm,
  footskip=15mm}

% Le titre
\title{Mon super rapport de stage}
\author{Alain Térieur}
\date{Année universitaire 2022--2023}

% Une macro
\newcommand{\oula}{(\textbf{???})}

% Macro pour page vide sans numérotation
\usepackage{afterpage}
\newcommand\myemptypage{
    \null
    \thispagestyle{empty}
    \addtocounter{page}{-1}
    \newpage
    }

% Packet pour les tableaux
\usepackage{array}
\usepackage{tabularx}

% Packet pour les tabulations
\usepackage{tabto}

%======================================================================
% D É B U T   D U   D O C U M E N T
%======================================================================

\begin{document}

  \thispagestyle{empty}
  \begin{center}
    \includegraphics[width=12cm]{Logo_IUT-UL_dept_info.png}
  \end{center}

  \vspace{1cm}

  \noindent
  {\large
    IUT Nancy Charlemagne\\
    Université de Lorraine\\
    2 ter boulevard Charlemagne\\
    BP 55227\\
    54052 Nancy Cedex\\[5mm]
    Département informatique
  }

  \vspace{5cm}

  \begin{center}
    {\huge
      \textbf{Synthèse des TD}
    }
  \end{center}

  \vspace{5cm}

  % \vspace{2cm}
  \vfill

  {\Large
    \noindent
    Etudiant : Théo Eisenbart\\
    Année universitaire 2022--2023
  }
  % Ajout d'une page vide
  \newpage
  \thispagestyle{empty}
  \mbox{}
  \newpage

  \newpage
  % Table des matières
  \tableofcontents

  \newpage

  \section{Efficacité de l'environnement de travail}

    \subsection{TD-1 : La souris}

  \begin{itemize}
    \item Désactiver votre souris au noveau système avec la commande xinput
  \end{itemize}

  \vspace{0.3cm}
  \$ sudo xinput disable\{2,4,6\}
  \vspace{0.3cm}

  \begin{itemize}
    \item Initialiser un fichier dans lequel nous allons lister tous les problèmes d'efficacité rencontrés pendant cette séance
  \end{itemize}
  \vspace{0.3cm}

  \begin{tabular}{|c|p{5cm}|p{10cm}|}
    \hline
    \textbf{Priorité} & \textbf{Problème} & \textbf{Correctif}\\
    \hline
    1 & Logout difficile au clavier & Raccourci clavier \textbf{Ctrl+Alt+Suppr/Delete}\\
    \hline
    2 & Impossible d'éditer des documents PDF avec Google Drive & Utilisation de LaTeX\\
    \hline
    3 & Impossible d'avoir plusieurs terminaux en parallèle & Sous Terminator, \textbf{Ctrl+Shift+O} pour split le terminal verticalement,\newline \textbf{Ctrl+Shift+L} pour split le terminal horizontalement\\
    \hline
    4 & Naviguer dans discord sans la souris & \textbf{Tab} pour se déplacer de haut en bas, \textbf{Shift+Tab} pour aller de bas en haut \newline et \textbf{Ctrl+Tab} pour naviguer de gauche à droite\\
    \hline
    5 & Se déplacer dans un éditeur de text sans la souris & Utilisation de Vim\\
    \hline
    6 & Ouvrir les settings de notre environnement graphique & shortcut configurable dans les settings ex : \textbf{Alt+S}\\
    \hline
    7 & Accèder au navigateur de fichier & shotcut configurable dans les settings ex : \textbf{Alt+F}\\
    \hline
  \end{tabular}

\vspace{0.3cm}
	\subsection{TD-2 : Le clavier}
  \begin{itemize}
	  \item Identifier un site permettant de s'améliorer à taper au clavier
  \end{itemize}

\vspace{0.3cm}
\href{https://10fastfingers.com/typing-test/french#}{Le site 10fastfingers} car :
 \vspace{0.3cm}
		\begin{itemize}
      \begin{itemize}
			  \item Il est gratuit
			  \item Il a de nombreuse langue à disposition
			  \item Il est très simple d'utilisation
      \end{itemize}
		\end{itemize}

\vspace{0.3cm}
\begin{itemize}
  \item Screen de l'interface :
\end{itemize}
\vspace{0.3cm}

\begin{center}
  \includegraphics[width=12cm]{screen-10fastfingers.png}
\end{center}

\vspace{0.3cm}
\begin{itemize}
  \item Quelque Résultat :
\end{itemize}

\thispagestyle{empty}
\begin{center}
  \includegraphics[width=4cm]{screen-resultat-1.png}\hfill
  \includegraphics[width=4cm]{screen-resultat-2.png}\hfill
  \includegraphics[width=4cm]{screen-resultat-3.png}
\end{center}

\vspace{0.3cm}
	\subsection{TD-3 : Le Shell}

\begin{itemize}
  \item Effectuer les tutoriels pour Vim et Emacs :
\end{itemize}
\vspace{0.3cm}

J'ai choisie d'apprendre VIM car c'est le plus utilisé et il est plus pratique que Emacs

\vspace{0.3cm}

\begin{itemize}
  \item Paramétrer GNU Readline :
  \vspace{0.3cm}

  \$ set -o emacs \newline
  ou \newline
  \$ set -o vi \newline
\end{itemize}

\begin{itemize}
  \item Tester les 2 modes, en choisir un :
\end{itemize}
\vspace{0.3cm}

J'ai choisie le mode Emacs car c'est le mode que j'ai le plus utilisé donc celui avec lequel je me sens le plus à l'aise
c'est par ailleurs le mode par défaut ce qui me facilite grandement si je dois travailler sur une autre machine

\vspace{0.3cm}

\begin{itemize}
  \item Paramétrer Emacs ou Vim comme étant éditeur par défaut en plus du mode de readline :

  \vspace{0.3cm}

  Ligne à ajouter dans le .bashrc .zshrc :
  \vspace{0.3cm} \newline
  \$ export EDITOR="vim"
  \vspace{0.3cm} \newline
  Ligne à effectuer dans le terminal pour chnager de readline :
  \vspace{0.3cm} \newline
  \$ set -o emacs \newline
  \newline
  J'ai donc choisie comme configuration "vim" en éditeur de texte par défaut et Emacs comme readline
\end{itemize}

  \subsection{TD-4 : Bash history}

\begin{itemize}
  \item Regardez votre "history" :

  \vspace{0.3cm}

  Voici ce qu'on obtient avec la commande : \$ history
\end{itemize}
\vspace{0.3cm}

\includegraphics[width=11cm]{screen-term-td4.png}
\vspace{0.3cm}

\begin{itemize}
  \item Y'a-t-il des informations sensibles ? Comment y rémédier ?

  \vspace{0.3cm}

  Oui il y a des informations sensibles car cette commande montre toutes les commandes qui ont été tapé (dans mon cas dans zsh) \newline
  Mais il y a un moyen d'y remédier, autorisé la lecture au fichier ".zsh\_history" uniquement à root
\end{itemize}

\end{itemize}



\end{document}
